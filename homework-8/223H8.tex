\documentclass[12pt, oneside]{amsart}
\usepackage{amsmath,amsfonts, amssymb, xcolor}
\usepackage{fullpage}
\newcommand{\Z}{\mathbb Z}
\newcommand{\R}{\mathbb R}
\DeclareMathOperator{\id}{id}
\DeclareMathOperator{\spn}{span}
\DeclareMathOperator{\proj}{proj}
\newcommand\norm[1]{\left\lVert#1\right\rVert}

\theoremstyle{definition}
\newtheorem{prob}{Problem}

\usepackage[most]{tcolorbox} % powerful colored boxes
\usepackage{xcolor}          % colors
\newtcolorbox{solution}{
  colframe=green!50!black,   % medium green border
  coltext=black,             % body text in black for readability
  boxrule=0.4pt,             % thin, professional border
  enhanced,                  % enable advanced drawing features
  left=6pt, right=6pt, top=6pt, bottom=6pt % padding
}

\title{Math 223 Fall 2025 - Homework 8}
\author{due November 16 at 11:59pm}
\pagenumbering{gobble}
\begin{document}
\maketitle
Each problem is worth 10 points.


\begin{prob}\item
\begin{enumerate}
    \item Prove the \emph{triangle inequality}:
$\norm{v+w}\leq \norm{v} + \norm{w}$ 
You can use other properties of the norms proven in class.

\begin{solution}

We start by computing the left side of the inequality:
\[
\norm{v+w}^2 = \langle v+w \mid v+w \rangle
\]

Using linearity in the first argument we can expand the inner product:
\[
= \langle v \mid v+w \rangle + \langle w \mid v+w \rangle
\]

Now using linearity in the second argument we can expand again:
\[
= \langle v \mid v \rangle + \langle v \mid w \rangle + \langle w \mid v \rangle + \langle w \mid w \rangle
\]

Since we are working over the reals, the inner product is symmetric, so $\langle v \mid w \rangle = \langle w \mid v \rangle$. Thus we can simplify the expression:
\[
= \norm{v}^2 + 2\langle v \mid w \rangle + \norm{w}^2
\]

By the Cauchy-Schwarz inequality, we know that $\langle v \mid w \rangle \leq \norm{v}\norm{w}$. Thus we can write our equation as an inequality:
\[
\norm{v+w}^2 = \norm{v}^2 + 2\langle v \mid w \rangle + \norm{w}^2 \leq \norm{v}^2 + 2\norm{v}\norm{w} + \norm{w}^2 = (\norm{v} + \norm{w})^2
\]

Taking square roots of both sides we get the original inequality we wanted to prove:
\[
\norm{v+w} \leq \norm{v} + \norm{w}
\]

\end{solution}

    \item Prove the \emph{parallelogram law}:
$\norm{v+w}^2+ \norm{v-w}^2 = 2 (\norm{v}^2 +\norm{w}^2)$

\begin{solution}

We start by expanding out the left side of the equation $\norm{v+w}^2 + \norm{v-w}^2$. First we expand the first term $\norm{v+w}^2$ and because we're working over the reals $\langle v \mid w \rangle = \langle w \mid v \rangle$ we can combine these terms as well:
\[
\norm{v+w}^2 = \langle v+w \mid v+w \rangle = \langle v\mid v \rangle + \langle v\mid w \rangle + \langle w\mid v \rangle + \langle w\mid w \rangle
\]

Now combining the middle two terms and writing the inner products of vectors with themselves as norms:
\[
\norm{v}^2 + 2\langle v\mid w \rangle + \norm{w}^2
\]

We can also expand out the second term of our original equation's left side $\norm{v-w}^2$:
\[
\norm{v-w}^2 = \langle v-w \mid v-w \rangle = \langle v\mid v \rangle - \langle v\mid w \rangle - \langle w\mid v \rangle + \langle w\mid w \rangle
\]

Combining terms again, using symmetry and the fact that we are working over the reals:
\[
\norm{v-w}^2 = \norm{v}^2 - 2\langle v\mid w \rangle + \norm{w}^2
\]

Finally we can combine the two expanded terms and rearrange to get the right side of the parallelogram law which we are trying to prove:
\[
\norm{v+w}^2 + \norm{v-w}^2 = (\norm{v}^2 + 2\langle v\mid w \rangle + \norm{w}^2) + (\norm{v}^2 - 2\langle v\mid w \rangle + \norm{w}^2)
\]

The inner product terms cancel from each and we get what we were intending to prove:
\[
= 2\norm{v}^2 + 2\norm{w}^2 = 2(\norm{v}^2 + \norm{w}^2) 
\]

So finally we have:
\[
\norm{v+w}^2 + \norm{v-w}^2 
= 2 (\norm{v}^2 + \norm{w}^2)
\]

\end{solution}

    \item What does the parallelogram law say about parallelograms in $\mathbb R^2$?
\end{enumerate}
\end{prob}

\begin{prob}
    Let $V$ be a real vectors space with an orthogonal basis $(v_1, \dots, v_n)$. Prove that for every $w\in V$ we have

    $$w = \proj_{v_1}(w) + \proj_{v_2}(w) + \dots \proj_{v_n}(w) = \frac{\langle v_1 \mid w \rangle}{\norm{v_1}^2} v_1+\frac{\langle v_2 \mid w \rangle}{\norm{v_2}^2} v_2 + \dots \frac{\langle v_n \mid w \rangle}{\norm{v_n}^2} v_n$$
\end{prob}

\begin{prob}\item
\begin{enumerate}
    \item 
    Apply the Gram-Schmidt process to the following basis $\mathcal B$ of $\mathbb R^3$ to obtain an orthonormal basis $\mathcal B'$ of $\mathbb R^3$
    $\mathcal B = \left(\left[\begin{matrix} 1 \\0 \\1
    \end{matrix}\right], \left[\begin{matrix} 0 \\1 \\1 
    \end{matrix}\right],\left[\begin{matrix} 1 \\3 \\3
    \end{matrix}\right] \right)$.

\begin{solution}

First, we normalize the first vector to get the first vector in our new basis:
\[
e_1 = \frac{v_1}{\norm{v_1}} = \frac{1}{\sqrt{2}}\begin{bmatrix} 1 \\ 0 \\ 1 \end{bmatrix} = \begin{bmatrix} 1/\sqrt{2} \\ 0 \\ 1/\sqrt{2} \end{bmatrix}
\]

For the remaining vectors, we subtract the projections onto the previous vectors in our new orthonormal set and then normalize them as well. 

For $u_2$:
\[
u_2 = v_2 - \proj_{e_1}(v_2)
\]

\[
= \begin{bmatrix} 0 \\ 1 \\ 1 \end{bmatrix} - \langle v_2, e_1 \rangle e_1 = \begin{bmatrix} 0 \\ 1 \\ 1 \end{bmatrix} - \frac{1}{\sqrt{2}}\begin{bmatrix} 1/\sqrt{2} \\ 0 \\ 1/\sqrt{2} \end{bmatrix} = \begin{bmatrix} -1/2 \\ 1 \\ 1/2 \end{bmatrix}
\]

Normalizing $e_2 = \frac{u_2}{\norm{u_2}} = \frac{2}{\sqrt{6}}\begin{bmatrix} -1/2 \\ 1 \\ 1/2 \end{bmatrix} = \begin{bmatrix} -1/\sqrt{6} \\ 2/\sqrt{6} \\ 1/\sqrt{6} \end{bmatrix}$

\vspace{0.3cm}

For $u_3$:
\[
u_3 = v_3 - \proj_{e_1}(v_3) - \proj_{e_2}(v_3)
\]

\[
= \begin{bmatrix} 1 \\ 3 \\ 3 \end{bmatrix} - \langle v_3, e_1 \rangle e_1 - \langle v_3, e_2 \rangle e_2 = \begin{bmatrix} 1 \\ 3 \\ 3 \end{bmatrix} - 2\sqrt{2} \begin{bmatrix} 1/\sqrt{2} \\ 0 \\ 1/\sqrt{2} \end{bmatrix} - \frac{4\sqrt{6}}{3}\begin{bmatrix} -1/\sqrt{6} \\ 2/\sqrt{6} \\ 1/\sqrt{6} \end{bmatrix} = \begin{bmatrix} 1/3 \\ 1/3 \\ -1/3 \end{bmatrix}
\]

Normalizing: $e_3 = \frac{u_3}{\norm{u_3}} = \sqrt{3}\begin{bmatrix} 1/3 \\ 1/3 \\ -1/3 \end{bmatrix} = \begin{bmatrix} 1/\sqrt{3} \\ 1/\sqrt{3} \\ -1/\sqrt{3} \end{bmatrix}$

Therefore, the orthonormal basis is the set of these three vectors:
\[
\mathcal{B}' = \left(\begin{bmatrix} 1/\sqrt{2} \\ 0 \\ 1/\sqrt{2} \end{bmatrix}, \begin{bmatrix} -1/\sqrt{6} \\ 2/\sqrt{6} \\ 1/\sqrt{6} \end{bmatrix}, \begin{bmatrix} 1/\sqrt{3} \\ 1/\sqrt{3} \\ -1/\sqrt{3} \end{bmatrix}\right)
\]

\end{solution}

    \item Find the coefficients of the vectors $w = \left[\begin{matrix} 1 \\1 \\2 
    \end{matrix}\right]$ in the basis $\mathcal B'$
\end{enumerate}
\end{prob}

\begin{prob}
Consider the inner product on the space $\mathcal P_2(\mathbb R)$ of polynomials of degree $\leq 2$ with real coefficients given by
$$
\langle f \mid g \rangle = \int_0^1 f(x)g(x) dx.
$$
\begin{enumerate}
    \item 
    Apply the Gram-Schmidt process to the basis $\mathcal B = (1,x, x^2)$ of $P_2(\mathbb R)$ to produce an orthonormal basis.    
    \item Find the coefficients of $h(x) = 1+x$ in the obtained basis.
\end{enumerate}
\end{prob}

\end{document}