\documentclass[12pt, oneside]{amsart}
\usepackage{amsmath,amsfonts, amssymb, xcolor}
\usepackage{fullpage}
\newcommand{\Z}{\mathbb Z}
\newcommand{\R}{\mathbb R}
\DeclareMathOperator{\id}{id}
\DeclareMathOperator{\spn}{span}

\usepackage[most]{tcolorbox} % powerful colored boxes
\usepackage{xcolor}          % colors
\newtcolorbox{solution}{
  colframe=green!50!black,   % medium green border
  coltext=black,             % body text in black for readability
  boxrule=0.4pt,             % thin, professional border
  enhanced,                  % enable advanced drawing features
  left=6pt, right=6pt, top=6pt, bottom=6pt % padding
}


\theoremstyle{definition}
\newtheorem{prob}{Problem}

\title{Math 223 Fall 2025 - Homework 6}
\author{due October 28 at 11:59pm}
\pagenumbering{gobble}
\begin{document}
\maketitle
Each problem is worth 10 points.

\begin{prob}
    The \emph{transpose} of an $m\times n$ matrix $A$ ($m$ rows, $n$ columns) is the $n\times m$ matrix $A^T$ ($n$ rows, $m$ columns) such that the columns of $A^T$ are the rows of $A$ (and so the rows of $A^T$ are the columns of $A$).

    \begin{enumerate}
        \item Compute the transpose of $A = \left[\begin{matrix}
        4 & 0 & 2 & 2 \\
        0 & 1 & 1 & 2
        \end{matrix}\right]$.
        
        \begin{solution}
        When we compute the transpose of a matrix, we swap the rows and columns. The columns of $A^T$ are the rows of $A$.
        
        \[
        A^T = \begin{bmatrix}
        4 & 0 \\
        0 & 1 \\
        2 & 1 \\
        2 & 2
        \end{bmatrix}
        \]
        \end{solution}
    \end{enumerate}
    Let $A$ be an $m\times n$ matrix and $B$ an $n\times k$ matrix.
    \begin{enumerate}
    \setcounter{enumi}{2}
    \item Check that matrices $(AB)^T$ and $B^T A^T$ have the same size.
    
    \begin{solution}
    
    First, we determine the size of $(AB)^T$:
    
    Since $A$ is $m \times n$ and $B$ is $n \times k$, the product $AB$ is an $m \times k$ matrix.
    
    When we transpose $AB$, the rows and columns are swapped. Therefore $(AB)^T$ is a $k \times m$ matrix.
    
    \vspace{0.3cm}
    
    Next, we determine the size of $B^T A^T$:
    
    Since $B$ is $n \times k$, its transpose $B^T$ is a $k \times n$ matrix.
    
    Since $A$ is $m \times n$, its transpose $A^T$ is an $n \times m$ matrix.
    
    The product $B^T A^T$ multiplies a $k \times n$ matrix by an $n \times m$ matrix resulting in a $k \times m$ matrix.
    
    \vspace{0.3cm}
    
    Finally, both $(AB)^T$ and $B^T A^T$ are $k \times m$ matrices, so they have the same size.
    \end{solution}
    
    \item Show that $(AB)^T = B^T A^T$. 
    (Describe how the $(i,j)$-entry of $(AB)^T$ is obtained from rows/columns of $A$ and $B$, and similarly describe how the $(i,j)$-entry of $B^T A^T$ is obtained, and compare these.)
    
    \begin{solution}
    We show that $(AB)^T = B^T A^T$ by comparing their $(i,j)$-entries $(AB)^T]_{i,j}$ and $[B^T A^T]_{i,j}$.
    
    \vspace{0.3cm}
    
    By definition of transpose:
    
    \[
    (AB)^T_{j,i} = (A B)_{i,j}
    \]
    
    The $(j,i)$-entry of $(AB)^T$ is computed by taking row $j$ of $A$ and dotting it with column $i$ of $B$:
    \[
    [(AB)^T]_{i,j} = [AB]_{j,i} = \sum_{\ell=1}^{n} A_{j,\ell} \cdot B_{\ell,i}
    \]
    
    \vspace{0.3cm}
    
    We compute the $(i,j)$-entry of $B^T A^T$ by taking row $i$ of $B^T$ and dotting it with column $j$ of $A^T$. 
    
    \hspace{2em} Row $i$ of $B^T$ = Column $i$ of $B$ = $(B_{1,i}, B_{2,i}, \ldots, B_{n,i})$

    \hspace{2em} Column $j$ of $A^T$ = Row $j$ of $A$ = $(A_{j,1}, A_{j,2}, \ldots, A_{j,n})$

    
    Therefore:
    $[B^T A^T]_{i,j} = \sum_{\ell=1}^{n} B_{\ell,i} \cdot A_{j,\ell}$
    
    \vspace{0.3cm}

    Since $A_{j,\ell}$ and $B_{\ell,i}$ are single elements in their respective matrices and thus are scalars, we know multiplication of these scalars is commutative. Thus:
    \[
      A_{j,\ell} \cdot B_{\ell,i} = B_{\ell,i} \cdot A_{j,\ell}
    \]
    \[
    \text{for each: } \ell
    \]
    
    Thus:
    \[
    [(AB)^T]_{i,j} = \sum_{\ell=1}^{n} A_{j,\ell} \cdot B_{\ell,i} = \sum_{\ell=1}^{n} B_{\ell,i} \cdot A_{j,\ell} = [B^T A^T]_{i,j}
    \]
    
    
    Because the $(i,j)$-entries are equal for all $i$ and $j$, we conclude that $(AB)^T = B^T A^T$. \qed
    \end{solution}

    \end{enumerate}

\end{prob}


\begin{prob} Let $A$ be an $n\times n$ matrix. 
    \begin{enumerate}
    \item Let $A = E_1\cdots E_\ell$ be the product of elementary matrices. Deduce from Problem 1 that $A^T = E_\ell^T \cdots E_1^T$.
    
    \begin{solution}
    
    
    We write our matrix product as: 
    
    \[
    A = E_1 E_2 \cdots E_\ell = (E_1 E_2 \cdots E_{\ell-1}) E_\ell
    \]

    Taking the transpose of both sides:
    
    \[
      A^T = (E_1 E_2 \cdots E_\ell)^T = ((E_1 E_2 \cdots E_{\ell-1}) E_\ell)^T
    \]
    
    Now we can apply the result of problem 1, $(AB)^T = B^T A^T$, with the product $(E_1 \cdots E_{\ell-1})$ as the first matrix and $E_\ell$ as the second matrix:
    
    
    \[
      A^T = E_\ell^T (E_1 \cdots E_{\ell-1})^T
    \]
    
    Repeatedly applying this step, we remove the rightmost $E_i$ from inside the parenthesis and places its transpose on the left. 

    After applying this to each element, $\ell-1$ times, we get the result:

    \[
      A^T = E_\ell^T E_{\ell-1}^T \cdots E_2^T E_1^T
    \]
    
    For example with $\ell = 3$:
    \[
      A = E_1 E_2 E_3 \implies A^T = ((E_1 E_2) E_3)^T = E_3^T (E_1 E_2)^T = E_3^T E_2^T E_1^T
    \]
    \end{solution}
    
    \item Show that for every elementary matrix $E$, $E^T$ is also an elementary matrix, and $\det E = \det E^T$.
    
    \begin{solution}
    
    Elementary matrices come from doing one of three row operations on the identity matrix. We will show that $E^T$ is an elementary matrix and $\det E = \det E^T$ for each of these three cases.
    
    \vspace{0.3cm}
    
    \textbf{Row swap operation}
    
    Consider the elementary matrix, $E$ where rows $i$ and $j$ are swapped from the identity matrix.
    
    Because the matrix transpose operation swaps the rows and columns of a matrix, when we transpose $E$ the operation of swapping rows $i$ and $j$ which we performed on the identity matrix, becomes swapping columns $i$ and $j$.
    
    Since the identity matrix is symmetric, swapping columns $i$ and $j$ of $I$ produces the same matrix as swapping rows $i$ and $j$ of $I$. Thus:
    \[
      E^T = E
    \]

    As these two matrices are equal, their determinants are equal:
    \[
    \det E = \det E^T
    \]
    
    \vspace{0.3cm}
    
    \textbf{Row scaling operation}
    
    Consider the elementary matrix, $E$, obtained by scaling row $i$ by $c \in F$ in the identity matrix.
    
    The matrix $E$ is diagonal as it is equal to the identity matrix with a single entry scaled. 

    Thus its entries along the diagonal are: $1, 1, \ldots, c, \ldots, 1$ where $c$ is in position $(i,i)$.
    
    Since $E$ is diagonal, transposing it produces the same matrix:
    \[
      E^T = E
    \]
    
    As these two matrices are equal, their determinants are equal:
    \[
    \det E = \det E^T
    \]
    
    \vspace{0.3cm}
    
    \textbf{Row addition operation}
    
    Consider the elementary matrix, $E$ obtained by adding $c$ times row $i$ to row $j$ in the identity matrix.
    
    The matrix $E$ has 1's on the diagonal, a $c$ in position $(j, i)$, and 0's elsewhere.
    
    When we transpose, the $c$ moves from position $(j, i)$ to position $(i, j)$. Therefore $E^T$ has 1's on the diagonal and a $c$ in position $(i, j)$.
    
    This means $E^T$ is equivelant to the identity matrix with the row operation of adding $c$ times row $j$ to row $i$. As that is a single row addition operation performed on the identity matrix, $E^T$ is an elementary matrix.
    
    We also know that row addition operations do not change the determinant, so:
    \[
    \det E = 1 = \det E^T
    \]
    
    \vspace{0.3cm}
    
    We have shown that in all three cases, $E^T$ is an elementary matrix and $\det E = \det E^T$. 
    \end{solution}
    
    \item Prove that for every $n\times n$ matrix $A$ we have $\det A = \det A^T$.
    \end{enumerate}
\end{prob}

\begin{prob} True or False. Justify your answers.
\begin{enumerate}
    \item  $\det (A+B) = \det A  +\det B$ where $A,B$ are $n\times n$ matrices.
    
    \begin{solution}
    False, which can be proved with the counter example:
    
    \[
    A = \begin{bmatrix} 1 & 0 \\ 0 & 0 \end{bmatrix}, \quad B = \begin{bmatrix} 0 & 0 \\ 0 & 1 \end{bmatrix}
    \]
    
    Finding the determinants of $A$ and $B$:
    \[
    \det A = 0 \quad \det B = 0
    \]
    \[
    \det A + \det B = 0 + 0 = 0
    \]
    
    However, when we add the matrices:
    \[
    A + B = \begin{bmatrix} 1 & 0 \\ 0 & 1 \end{bmatrix}
    \]
    
    \[
    \det(A+B) = 1 
    \]

    Finally:
  \[
  \det(A+B) \neq 0 = \det A + \det B
  \]

    \end{solution}
    
    \item $\det c\cdot A = c^n \det A$ where $A$ is an $n\times n$ matrix, $c\in F$ and $c\cdot A$ is obtained from $A$ by multiplying each entry by $c$.
    
    \begin{solution}
    We know from the properties of determinants that multiplying a single row by a scalar $c$ multiplies the determinant by $c$.
    
    When we compute $c \cdot A$, we are multiplying every entry of $A$ by $c$, which is the same as multiplying each of the $n$ rows by $c$.
    
    Since each time we multiply a row by $c$, this multiplies the determinant by $c$, and we perform this operation $n$ times (once for each row), the determinant is multiplied by $c$ a total of $n$ times.

    Thus:
    \[
    \det(c \cdot A) = c^n \det A
    \]
    
    Therefore the statement is true.
    \end{solution}
\end{enumerate}
\end{prob}

\begin{prob} Compute the determinant.
\begin{enumerate}
    \item $A_1= \left[\begin{matrix}
        1 & 0 & -2 & 3 \\
        -3 & 1 & 1 & 2 \\
        0 & 4 & -1 & 1 \\
        2 & 3 & 0 & 1
    \end{matrix}\right]$\\\\
    
    \begin{solution}
    We compute the determinant via row reduction to an upper triangular matrix and then take the product of the diagonal:
    
    Row Operations:
    
    $R_2 = R_2 + 3R_1$
    
    $R_4 = R_4 - 2R_1$
    
    $R_3 = R_3 - 4R_2$
    
    $R_4 = R_4 - 3R_2$
    
    $R_4 = R_4 - R_3$

    This results in the upper triangular matrix:
    
    \[
    \begin{bmatrix}
    1 & 0 & -2 & 3 \\
    0 & 1 & -5 & 11 \\
    0 & 0 & 19 & -43 \\
    0 & 0 & 0 & 5
    \end{bmatrix}
    \]
    
    The determinant is the product of the diagonal entries:
    \[
    \det A_1 = 1 \cdot 1 \cdot 19 \cdot 5 = 95
    \]
    \end{solution}
    
    \item $A_2= \left[\begin{matrix}
        -1 & 3 & 2 \\4 & -8 & 1 \\
        2 & 2 & 5
    \end{matrix}\right]$\\\\
    
    \begin{solution}
    We use the known formula for a cofactor expansion for a 3×3 matrix, which gives the determinant:
    
    \[
    \det A_2 = -1 \begin{vmatrix} -8 & 1 \\ 2 & 5 \end{vmatrix} - 3 \begin{vmatrix} 4 & 1 \\ 2 & 5 \end{vmatrix} + 2 \begin{vmatrix} 4 & -8 \\ 2 & 2 \end{vmatrix}
    \]
    
    \[
    = -1[(-8)(5) - (1)(2)] - 3[(4)(5) - (1)(2)] + 2[(4)(2) - (-8)(2)]
    \]
    
    \[
    = -1[-42] - 3[18] + 2[24] = 42 - 54 + 48 = 36
    \]
    
    Therefore $\det A_2 = 36$.
    \end{solution}

    \item $A_3= \left[\begin{matrix}
        0 & 0 & 0 & 0 & 0 & -1 \\
        0 & 0 & 0 & 0 & 2 & 0 \\ 
        0 & 0 & 0 & 1 & 0 & 0 \\
        0 & 0 & 3 & 0 & 0 & 0 \\
        0 & 4 & 0 & 0 & 0 & 0 \\
        1 & 0 & 0 & 0 & 0 & 0 
    \end{matrix}\right]$
    
    \begin{solution}
    We compute the determinant via row reduction to a diagonal matrix and then take the product of the diagonal:
    
    Row Operations:
    
    Swap: $R_1$ and $R_6$

    Swap: $R_2$ and $R_5$

    Swap: $R_3$ and $R_4$

    This results in the upper triangular matrix:
    \[
    \begin{bmatrix}
    1 & 0 & 0 & 0 & 0 & 0 \\
    0 & 4 & 0 & 0 & 0 & 0 \\
    0 & 0 & 3 & 0 & 0 & 0 \\
    0 & 0 & 0 & 1 & 0 & 0 \\
    0 & 0 & 0 & 0 & 2 & 0 \\
    0 & 0 & 0 & 0 & 0 & -1
    \end{bmatrix}
    \]
    
    The product of the diagonal entries is:
    \[
    1 \cdot 4 \cdot 3 \cdot 1 \cdot 2 \cdot (-1) = -24
    \]
    
    We performed 3 row swaps, and each swap multiplies the determinant by $-1$:
    \[
    \det A_3 = (-1)^3 \cdot (-24) = 24
    \]
    \end{solution}
\end{enumerate}
\end{prob}

\end{document}