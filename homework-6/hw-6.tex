\documentclass[12pt, oneside]{amsart}
\usepackage{amsmath,amsfonts, amssymb, xcolor}
\usepackage{fullpage}
\newcommand{\Z}{\mathbb Z}
\newcommand{\R}{\mathbb R}
\DeclareMathOperator{\id}{id}
\DeclareMathOperator{\spn}{span}

\usepackage[most]{tcolorbox} % powerful colored boxes
\usepackage{xcolor}          % colors
\newtcolorbox{solution}{
  colframe=green!50!black,   % medium green border
  coltext=black,             % body text in black for readability
  boxrule=0.4pt,             % thin, professional border
  enhanced,                  % enable advanced drawing features
  left=6pt, right=6pt, top=6pt, bottom=6pt % padding
}


\theoremstyle{definition}
\newtheorem{prob}{Problem}

\title{Math 223 Fall 2025 - Homework 6}
\author{due October 28 at 11:59pm}
\pagenumbering{gobble}
\begin{document}
\maketitle
Each problem is worth 10 points.

\begin{prob}
    The \emph{transpose} of an $m\times n$ matrix $A$ ($m$ rows, $n$ columns) is the $n\times m$ matrix $A^T$ ($n$ rows, $m$ columns) such that the columns of $A^T$ are the rows of $A$ (and so the rows of $A^T$ are the columns of $A$).

    \begin{enumerate}
        \item Compute the transpose of $A = \left[\begin{matrix}
        4 & 0 & 2 & 2 \\
        0 & 1 & 1 & 2
        \end{matrix}\right]$.
    \end{enumerate}
    Let $A$ be an $m\times n$ matrix and $B$ an $n\times k$ matrix.
    \begin{enumerate}
    \setcounter{enumi}{2}
    \item Check that matrices $(AB)^T$ and $B^T A^T$ have the same size.
    \item Show that $(AB)^T = B^T A^T$. 
    (Describe how the $(i,j)$-entry of $(AB)^T$ is obtained from rows/columns of $A$ and $B$, and similarly describe how the $(i,j)$-entry of $B^T A^T$ is obtained, and compare these.)
    
    \begin{solution}
    We will show that $(AB)^T$ and $B^T A^T$ have the same $(i,j)$-entry for all $i,j$.
    
    \vspace{0.3cm}
    \noindent\textbf{The $(i,j)$-entry of $(AB)^T$:} 
    
    By definition of transpose, the $(i,j)$-entry of $(AB)^T$ equals the $(j,i)$-entry of $AB$. 
    
    The $(j,i)$-entry of $AB$ is computed by taking row $j$ of $A$ and dotting it with column $i$ of $B$:
    $[(AB)^T]_{i,j} = [AB]_{j,i} = \sum_{\ell=1}^{n} A_{j,\ell} \cdot B_{\ell,i}$
    
    \vspace{0.3cm}
    \noindent\textbf{The $(i,j)$-entry of $B^T A^T$:} 
    
    This is computed by taking row $i$ of $B^T$ and dotting it with column $j$ of $A^T$. 
    
    Note that:
    \begin{itemize}
        \item Row $i$ of $B^T$ = Column $i$ of $B$ = $(B_{1,i}, B_{2,i}, \ldots, B_{n,i})$
        \item Column $j$ of $A^T$ = Row $j$ of $A$ = $(A_{j,1}, A_{j,2}, \ldots, A_{j,n})$
    \end{itemize}
    
    Therefore:
    $[B^T A^T]_{i,j} = \sum_{\ell=1}^{n} B_{\ell,i} \cdot A_{j,\ell}$
    
    \vspace{0.3cm}
    \noindent\textbf{Comparison:} 
    
    Since multiplication of scalars is commutative, we have 
    $A_{j,\ell} \cdot B_{\ell,i} = B_{\ell,i} \cdot A_{j,\ell}$
    for each $\ell$. 
    
    Thus:
    $[(AB)^T]_{i,j} = \sum_{\ell=1}^{n} A_{j,\ell} \cdot B_{\ell,i} = \sum_{\ell=1}^{n} B_{\ell,i} \cdot A_{j,\ell} = [B^T A^T]_{i,j}$
    
    Since the $(i,j)$-entries are equal for all $i$ and $j$, we conclude that $(AB)^T = B^T A^T$. \qed
    \end{solution}

    \end{enumerate}

\end{prob}


\begin{prob} Let $A$ be an $n\times n$ matrix. 
    \begin{enumerate}
    \item Let $A = E_1\cdots E_\ell$ be the product of elementary matrices. Deduce from Problem 1 that $A^T = E_\ell^T \cdots E_1^T$.
    
    \item Show that for every elementary matrix $E$, $E^T$ is also an elementary matrix, and $\det E = \det E^T$.
    
    \item Prove that for every $n\times n$ matrix $A$ we have $\det A = \det A^T$.
    \end{enumerate}
\end{prob}

\begin{prob} True or False. Justify your answers.
\begin{enumerate}
    \item  $$\det (A+B) = \det A  +\det B$$ where $A,B$ are $n\times n$ matrices.
    
    \item $$\det c\cdot A = c^n \det A$$ where $A$ is an $n\times n$ matrix, $c\in F$ and $c\cdot A$ is obtained from $A$ by multiplying each entry by $c$.
\end{enumerate}
\end{prob}

\begin{prob} Compute the determinant.
\begin{enumerate}
    \item $A_1= \left[\begin{matrix}
        1 & 0 & -2 & 3 \\
        -3 & 1 & 1 & 2 \\
        0 & 4 & -1 & 1 \\
        2 & 3 & 0 & 1
    \end{matrix}\right]$\\\\
    
    \item $A_2= \left[\begin{matrix}
        -1 & 3 & 2 \\4 & -8 & 1 \\
        2 & 2 & 5
    \end{matrix}\right]$\\\\

    \item $A_3= \left[\begin{matrix}
        0 & 0 & 0 & 0 & 0 & -1 \\
        0 & 0 & 0 & 0 & 2 & 0 \\ 
        0 & 0 & 0 & 1 & 0 & 0 \\
        0 & 0 & 3 & 0 & 0 & 0 \\
        0 & 4 & 0 & 0 & 0 & 0 \\
        1 & 0 & 0 & 0 & 0 & 0 
    \end{matrix}\right]$
\end{enumerate}
\end{prob}

\end{document}