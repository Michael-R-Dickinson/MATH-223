\documentclass[12pt]{article}
\usepackage{amsmath,amsfonts, amssymb}
\usepackage{fullpage}

\title{MATH 223 Homework Style Guide}
\author{Style Rules for Solution Writing}
\date{}

\begin{document}
\maketitle

\section*{Overall Philosophy}
Write solutions that demonstrate understanding while learning proof techniques. The goal is clear communication, not perfection. Show your work and reasoning in a way that someone at your level could follow.

\section{Mathematical Writing Style}

\subsection{Proof Structure}
\begin{itemize}
\item \textbf{Step-by-step approach}: Break complex arguments into digestible steps with clear transitions like ``First, we determine...'', ``Next, we...'', ``Finally...''
\item \textbf{Case-by-case analysis}: When multiple cases exist, do not use bold headings for the case names
\item \textbf{Vertical spacing}: Use \verb|\vspace{0.3cm}| to separate major thoughts or case divisions
\item \textbf{Concluding statements}: End proofs with explicit summary sentences like ``We have shown that...'' or ``Therefore...''
\end{itemize}

\subsection{Explanatory Approach}
\begin{itemize}
\item \textbf{Explain, don't just compute}: Before diving into algebra, briefly explain what you're about to do
\item \textbf{State intermediate results}: Don't skip steps - show how you get from A to B to C
\item \textbf{Justify non-obvious claims}: If you use a property (like commutativity), explicitly state it: ``Since $A_{j,\ell}$ and $B_{\ell,i}$ are single elements... multiplication of these scalars is commutative''
\item \textbf{Be somewhat redundant}: It's okay to over-explain when learning - better to be clear than terse
\end{itemize}

\subsection{Transitions and Flow}
Common transition phrases that feel natural:
\begin{itemize}
\item ``Now we can apply...''
\item ``We can do the same...''
\item ``Thus...'' (use frequently between steps)
\item ``As these two matrices are equal...''
\item ``Repeatedly applying this step...''
\item ``This means...''
\item ``We also know that...''
\end{itemize}

Avoid overly formal or stilted transitions - write conversationally but precisely.

\section{Notation and Display Math}

\subsection{Display Math Usage}
\begin{itemize}
\item Put important equations and results on their own lines using displaymath
\item Don't be afraid to show intermediate algebraic steps each on their own line
\item Use inline math for brief expressions within sentences
\item When listing operations or row operations, can use either display math or just text
\end{itemize}

\subsection{Matrix Notation}
\begin{itemize}
\item Use \verb|\begin{bmatrix}...\end{bmatrix}| for matrices
\item For 2x2 determinants in cofactor expansion, use \verb|\begin{vmatrix}...\end{vmatrix}|
\item Entry notation: use subscripts like $A_{i,j}$ or $[AB]_{i,j}$
\item Be consistent within a single problem but not necessarily across the whole assignment
\end{itemize}

\subsection{Special Notation}
\begin{itemize}
\item Use \verb|\qed| sparingly and inconsistently (you're still learning when proofs are ``done'')
\item Use \verb|\hspace{2em}| for indented sub-explanations
\item List row operations explicitly, either as ``$R_2 = R_2 + 3R_1$'' or as plain text
\end{itemize}

\section{Problem-Specific Approaches}

\subsection{Computational Problems}
\begin{itemize}
\item Show the setup briefly, then compute
\item For determinants: state your method (``row reduction'', ``cofactor expansion'', etc.)
\item Show intermediate matrices when helpful
\item State final answer clearly at the end
\end{itemize}

\subsection{True/False Questions}
\begin{itemize}
\item State ``True'' or ``False'' immediately
\item For false: give a concrete counterexample with small matrices
\item Show the computation that demonstrates the contradiction
\item For true: give a brief proof or cite a known property with explanation
\end{itemize}

\subsection{Proof Problems}
\begin{itemize}
\item Break into cases when natural (invertible vs. non-invertible, different types of operations)
\item Show how general facts apply to your specific situation
\item Give examples when it helps illustrate a pattern (``For example with $\ell = 3$...'')
\item Don't be afraid to spell out ``obvious'' steps
\end{itemize}

\section{Things That Make Writing Feel Human}

\subsection{Acceptable Imperfections}
\begin{itemize}
\item Slight inconsistencies in notation across problems are fine
\item Varying levels of detail (more detail when you're less confident)
\item Occasionally awkward phrasing (``for each: $\ell$'')
\item Some redundancy in explanations
\end{itemize}

\subsection{Signs of Learning}
\begin{itemize}
\item Over-justification of simple steps (shows you're being careful)
\item Explicitly stating properties you use (commutativity, determinant rules)
\item Breaking things into more steps than strictly necessary
\item Using phrases like ``we know from...'' to reference earlier work
\item Occasionally verbose explanations
\end{itemize}

\subsection{Avoid}
\begin{itemize}
\item Don't be overly formal or use unnecessarily advanced terminology
\item Don't apologize or express uncertainty (``I think...'', ``Maybe...'')
\item Avoid meta-commentary (``This proof will...'', ``The key insight is...'')
\item Don't use bullet points in proofs (use prose with display math)
\item Don't start every sentence the same way
\end{itemize}

\section{Formatting Conventions}

\begin{itemize}
\item Use \verb|\vspace{0.3cm}| between major sections of a solution
\item Bold case headings or major divisions
\item Use proper theorem environments (the template handles this)
\item Put final answers on their own line when appropriate
\item Use \verb|$\implies$| for logical implication, not ``$\rightarrow$''
\end{itemize}

\section{Final Reminders}

\begin{enumerate}
\item Clarity over elegance - you're learning, not writing for publication
\item Show your reasoning - partial credit matters
\item Be consistent within each problem but variation across problems is natural
\item Computational problems can be more terse; proof problems need more explanation
\item When in doubt, explain more rather than less
\item Your writing should sound like you understand the material, not like you memorized it
\end{enumerate}

\end{document}
