\documentclass[12pt, oneside]{amsart}
\usepackage{amsmath,amsfonts, amssymb}
\usepackage{fullpage}
\usepackage{enumitem}
\newcommand{\Z}{\mathbb Z}

\usepackage[most]{tcolorbox} % powerful colored boxes
\usepackage{xcolor}          % colors

% Solution box defined using generative AI:
\newtcolorbox{solution}{
  colframe=green!50!black,   % medium green border
  coltext=black,             % body text in black for readability
  boxrule=0.4pt,             % thin, professional border
  enhanced,                  % enable advanced drawing features
  left=6pt, right=6pt, top=6pt, bottom=6pt % padding
}


\title{Math 223 Homework 1}
\author{due September 16, 2025 at 11:59pm}
\pagenumbering{gobble}
\begin{document}
\maketitle
Each problem is worth 10 points.\\

\begin{enumerate}[label=\underline{Problem \arabic*}:]
\item  Let $F$ be a field. Derive the following statements directly from the axioms of a field. You can also refer to the facts that we proved in class (i.e.\ $a\cdot 0 = 0$ for all $a\in F$, and the cancellation law).
\begin{enumerate}
    \item For every $a\in F$, its additive inverse is unique. (Hint: assume $b,c\in F$ both satisfy the property of being the additive inverse of $a$, and deduce that $b=c$.)
    
\begin{solution}
Assume that there exist a $b$ and $b'$ that are both additive inverses of $a$:
\[
\exists b,b' \in F : a + b = 0 \;\land \:a + b' = 0
\]
As the right side of both equations are equal to 0, we can set the two equations equal:
\[
a + b = a + b'
\]
We have already proved the cancellation rule for equations in a field:
\[
a+b=a+c \implies b=c
\]
Therefore we can cancel the $a$'s in the equation and we are left with the equation:
\[
b=b'
\]
This implies that any two additive inverses of $a$ are in fact the same element. Thus each additive inverse of $a$ is unique.

\end{solution}


    \item For every $a,b\in F$, we have $(-a)\cdot b = -(a\cdot b)$\\

\begin{solution}
Consider the equation:
\[
(-a)\cdot b + a\cdot b
\]


We can use the distributive property over addition to simplify the expression:
\[
b\cdot (-a + a)
\]
Because $-a$ is the additive inverse of $a$, their sum is 0. 
And, because the product of zero and any element of the field is 0, the entire expression evaluates to 0:
\[
b\cdot(-a + a)
= b\cdot 0
= 0
\]

Tracing this result back to our original expression:
\[
(-a)\cdot b + (b \cdot a) = 0
\]
Because this sum evaluates to $0$ it is clear that $(-a)\cdot b$ is the additive inverse of $a \cdot b$.

Because we established in the previous proof that additive inverses are unique and we know that $-(a\cdot b)$ is also the additive inverse of $a\cdot b$ by the definition of additive inverses, $-(a\cdot b)$ must be equal to $(-a)\cdot b$.


\end{solution}


\end{enumerate}

\noindent Below are more facts that are true in every field. You can prove them for practice, but do not submit those proofs. They are proven in the appendix C of Freidberg-Insel-Spence.
Please make sure that you understand their statements. From now on, you can use those facts without proving them.
\begin{enumerate}[label=(\alph*), start=3]
    \item $0$ is unique.
    \item $1$ is unique.
    \item For every $a\in F$ such that $a\neq 0$, its multiplicative inverse is unique.
    \item $0$ has no multiplicative inverse.
    \item For every $a\in F$ we have $-(-a) = a$ (i.e.\ the additive inverse of the additive inverse of $a$ is $a$).
    \item For every non-zero $a\in F$ we have $(a^{-1})^{-1} = a$.
    \item For every $a, b,c \in F$ where $a\neq 0$ $$a\cdot b = a\cdot c\implies b=c.$$
    \item For every $a,b\in F$, if $a\cdot b = 0$, then $a=0$ or $b=0$.\\
\end{enumerate}

\item (Integers mod $n$) 
Let $a,b,c,n\in \mathbb Z$. We say $a$ and $b$ are \emph{congurent mod n} and write $ a \equiv b \!\!\mod n$, if $a = c\cdot n + b$.

For every natural number $n\geq 1$ we define $\mathbb Z_n$ as the set $\{0,1,2,\dots, n-1\}$ where the addition and multiplication are taken mod $n$.
For example, in $\mathbb Z_5$,
\begin{align*}
2 + 4 &\equiv 1 \!\!\mod 5  &(\text{since }2+4 = 6 \equiv 1 \mod 5)\\
4\cdot 3 &\equiv 2\!\!\mod 5 &(\text{since }4\cdot 3 = 12  \equiv 2 \mod 5)\\
(-2) &\equiv 3 \!\! \mod 5& (\text{since }2 + 3 = 5 \equiv 0 \mod 5)\\
\end{align*}

\begin{enumerate}
\item For each elements in $\mathbb Z_5$, list its additive inverse.

\begin{solution}
Additive inverses separated by ":":
 
1: 4

2: 3

3: 2

4: 1

0: 0
\end{solution}

\item For each \underline{non-zero} elements in $\mathbb Z_5$, list its multiplicative inverse.

\begin{solution}
Multiplicative inverses separated by ":":

1: 1

2: 3

3: 2

4: 4

\end{solution}

Convince yourself that $\mathbb Z_5$ is a field (you do not need to write it down).

\item Prove that $\mathbb Z_4$ is not a field.\\
\end{enumerate}
\begin{solution}
    Assume for contradiction that $\mathbb Z_4$ is a field. Thus, every non-zero element has a multiplicative inverse satisfying the equation:
    \[
        a \cdot a^{-1} = 1
    \]
    So in this field, 2 must have a multiplicative inverse $a$ (and 2 is in the field because it is between 0 and $4-1=3$):
    \[
        2\cdot a \equiv 1 \mod{4}
    \]
    Multiply both sides by 2:
    \[
        2\cdot a \equiv 1 \bmod{4} \implies 
        4a \equiv 2 \bmod{4}
    \]


    For an integer $a$, in $\mathbb Z_4$ 
    \[
        4a \equiv 0 \mod{4}
    \]
    Because $4a$ is divisible by 4 for every integer $a$.

    
    However we already had that 
    \[
        4a \equiv 2 \mod{4}
    \]
    Thus we have contradiction. Therefore $\mathbb Z_4$ is not a field. 
    
\end{solution}

\noindent \emph{Bonus} (not graded, do not submit): Prove that $\mathbb Z_n$ is a field if and only if $n$ is a \emph{prime} number (i.e.\ $n$ is a natural number greater than $1$ and its only divisors are $1$ and $n$).\\

\item Computational practice with vectors in $\mathbb R^n$: \\
\begin{enumerate}
    \item Find a vector $\vec{v} = (v_1, v_2, v_3, v_4)$ in $\mathbb R^4$ such that $$(2,1,7,-11) + 2\vec{v} = (4,3,1,-3)$$

    \begin{solution}
        $$
        \vec{v}=(1,1,-3,4)
        $$
    \end{solution}
    \item Does there exist $a\in \mathbb R$ such that this vector equality holds in $\mathbb R^3$? Explain your answer.
    $$
    a(2,-1,0) = (4,2,5)
    $$
    \begin{solution}
        There is no $a \in \mathbb R$ such that the equality holds in $\mathbb R^3$. One reason is that the third element in vector $(2,-1,0)$ is $0$ and no scalar multiple will make it equal to its corresponding element in $(4,2,5)$, 5.
    \end{solution}
\end{enumerate}
\end{enumerate}

\end{document}