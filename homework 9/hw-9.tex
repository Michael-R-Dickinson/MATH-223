\documentclass[12pt, oneside]{amsart}
\usepackage{amsmath,amsfonts, amssymb, xcolor}
\usepackage{fullpage}

\usepackage[most]{tcolorbox} % powerful colored boxes
\usepackage{xcolor}          % colors
\newtcolorbox{solution}{
    breakable,                 % allow page breaks inside the box
  colframe=green!50!black,   % medium green border
  coltext=black,             % body text in black for readability
  boxrule=0.4pt,             % thin, professional border
  enhanced,                  % enable advanced drawing features
  left=6pt, right=6pt, top=6pt, bottom=6pt % padding
}

\newcommand{\Z}{\mathbb Z}
\newcommand{\R}{\mathbb R}
\DeclareMathOperator{\id}{id}
\DeclareMathOperator{\spn}{span}
\DeclareMathOperator{\proj}{proj}
\newcommand\norm[1]{\left\lVert#1\right\rVert}

\theoremstyle{definition}
\newtheorem{prob}{Problem}

\title{Math 223 Fall 2025 - Homework 9}
\author{due November 25 at 11:59pm}
\pagenumbering{gobble}
\begin{document}
\maketitle
Each problem is worth 10 points.



\begin{prob}
Let $V =\mathbb R^n$ with the standard inner product (the dot product). Find $\text{proj}_{U}(v)$ where
\begin{enumerate}
    \item $n= 2$, $U = \{(x,y)\in \mathbb R^2\mid y=4x\}$, and $v = \left[\begin{matrix} 2\\6\end{matrix}\right]$.
    \item $n=3$, $U = \{(x,y,z)\in \mathbb R^3\mid x+3y-2z=0\}$, and $v = \left[\begin{matrix} 2\\1\\3\end{matrix}\right]$.
\end{enumerate}
\end{prob}

\begin{prob}
    Let $U = \spn \left(\left[\begin{matrix} 1\\2\\3\\-4\end{matrix}\right], \left[\begin{matrix} -5\\4\\3\\2\end{matrix}\right]\right)$ be a subspace of $\mathbb R^4$. Find an orthonormal basis of $U$ and an orthonormal basis of $U^{\perp}$.
\end{prob}

\begin{solution}
To find an orthonormal basis of $U$ we take our existing basis for $U$ and apply the Gram-Schmidt procedure. We start with $u_1$ and normalize it, then compute all remaining orthonormal vectors by subtracing the projections onto the previous vectors and normalizing.

First we normalize $u_1$ to get our first orthonormal vector $e_1$:
\[
\norm{u_1}^2 = 1 + 4 + 9 + 16 = 30
\]

\[
e_1 = \frac{1}{\sqrt{30}} \begin{bmatrix} 1\\2\\3\\-4\end{bmatrix}.
\]

Next we get an orthogonal vector $w_2$ by subtracing the component of $u_1$ from $u_2$ to orthagonalize it:

\[
w_2 = u_2 - \frac{\langle u_2, u_1 \rangle}{\norm{u_1}^2} u_1 = u_2 - \frac{4}{30} u_1 = u_2 - \frac{2}{15} u_1.
\]

\[
\langle u_2, u_1 \rangle = 4.
\]

\[
w_2 = \begin{bmatrix} -5 - 2/15 \\ 4 - 4/15 \\ 3 - 6/15 \\ 2 + 8/15 \end{bmatrix} =
\frac{1}{15} \begin{bmatrix} -77 \\ 56 \\ 39 \\ 38 \end{bmatrix}.
\]

Finally we normalize $w_2$ to get $e_2$:
\[
e_2 = \frac{w_2}{\norm{w_2}}
\]

\[
\norm{w_2}^2
= \frac{1}{225}(77^2 + 56^2 + 39^2 + 38^2) 
= \frac{802}{15}
\]
\[
\norm{w_2} = \sqrt{802/15} = \frac{\sqrt{12030}}{15}
\]

\[
e_2 = \frac{w_2}{\norm{w_2}} = \frac{1}{\sqrt{12030}} \begin{bmatrix} -77 \\ 56 \\ 39 \\ 38 \end{bmatrix}.
\]

Finally, we have applied the Grahm Schmidt procedure to all vectors in our bass to obtain an orthonormal basis of $U$. Our basis is the set of vectors $(e_1, e_2)$:
\[
\left( \frac{1}{\sqrt{30}} \begin{bmatrix} 1\\2\\3\\-4\end{bmatrix}, \frac{1}{\sqrt{12030}} \begin{bmatrix} -77\\56\\39\\38\end{bmatrix} \right).
\]

\vspace{0.3cm}

Now we find an orthonormal basis of $U^\perp$. The orthogonal complement $U^\perp$ consists of all vectors in $\R^4$ that are orthogonal to both $u_1$ and $u_2$. We set up the system
\[
\begin{bmatrix} 1 & 2 & 3 & -4 \\ -5 & 4 & 3 & 2 \end{bmatrix} \begin{bmatrix} x_1 \\ x_2 \\ x_3 \\ x_4 \end{bmatrix} = \begin{bmatrix} 0 \\ 0 \end{bmatrix}
\]
where the rows are the spanning vectors of $U$. We row reduce by applying $R_2 \to R_2 + 5R_1$:
\[
\begin{bmatrix} 1 & 2 & 3 & -4 \\ 0 & 14 & 18 & -18 \end{bmatrix}.
\]
From the second row, $14x_2 + 18x_3 - 18x_4 = 0$, which gives $x_2 = \frac{9(x_4 - x_3)}{7}$. Substituting into the first row, we get
\[
x_1 = -2x_2 - 3x_3 + 4x_4 = -\frac{18(x_4 - x_3)}{7} - 3x_3 + 4x_4 = \frac{10x_4 - 3x_3}{7}.
\]
Setting $(x_3, x_4) = (7, 0)$ gives $x_1 = -3$ and $x_2 = -9$, so $v_1 = \begin{bmatrix} -3 \\ -9 \\ 7 \\ 0 \end{bmatrix}$.

Setting $(x_3, x_4) = (0, 7)$ gives $x_1 = 10$ and $x_2 = 9$, so $v_2 = \begin{bmatrix} 10 \\ 9 \\ 0 \\ 7 \end{bmatrix}$.

Thus $(v_1, v_2)$ is a basis for $U^\perp$. We now apply Gram--Schmidt to obtain an orthonormal basis.

\vspace{0.3cm}

First, we normalize $v_1$. We have
\[
\norm{v_1}^2 = 9 + 81 + 49 + 0 = 139
\]
so we set
\[
e_1 = \frac{1}{\sqrt{139}} \begin{bmatrix} -3 \\ -9 \\ 7 \\ 0 \end{bmatrix}.
\]

Next, we orthogonalize $v_2$ against $v_1$. We compute
\[
\langle v_2, v_1 \rangle = (10)(-3) + (9)(-9) + (0)(7) + (7)(0) = -30 - 81 = -111.
\]
Thus
\[
w_2 = v_2 - \frac{\langle v_2, v_1 \rangle}{\norm{v_1}^2} v_1 = v_2 - \frac{-111}{139} v_1 = v_2 + \frac{111}{139} v_1.
\]
Computing each entry, we get
\[
w_2 = \begin{bmatrix} 10 - 333/139 \\ 9 - 999/139 \\ 777/139 \\ 7 \end{bmatrix} = \frac{1}{139} \begin{bmatrix} 1057 \\ 252 \\ 777 \\ 973 \end{bmatrix} = \frac{7}{139} \begin{bmatrix} 151 \\ 36 \\ 111 \\ 139 \end{bmatrix}.
\]

Now we normalize $w_2$. We have
\[
\norm{w_2}^2 = \frac{49}{139^2}(151^2 + 36^2 + 111^2 + 139^2) = \frac{49}{139^2}(22801 + 1296 + 12321 + 19321) = \frac{49 \cdot 55739}{139^2}
\]
so $\norm{w_2} = \frac{7\sqrt{55739}}{139}$. Thus
\[
e_2 = \frac{w_2}{\norm{w_2}} = \frac{1}{\sqrt{55739}} \begin{bmatrix} 151 \\ 36 \\ 111 \\ 139 \end{bmatrix}.
\]

Therefore, an orthonormal basis of $U^\perp$ is
\[
\left( \frac{1}{\sqrt{139}} \begin{bmatrix} -3\\-9\\7\\0\end{bmatrix}, \frac{1}{\sqrt{55739}} \begin{bmatrix} 151\\36\\111\\139\end{bmatrix} \right).
\]
\end{solution}

\begin{prob}
Let $V$ be a finite dimensional inner product space. 
\begin{enumerate}
    \item Let $U\subseteq V$ a subspace with a basis $(u_1,\dots, u_k)$. Show that if we complete $(u_1,\dots, u_k)$ to a basis $(u_1,\dots, u_k, u_{k+1}, \dots, u_n)$ of $V$ and apply the Gram-Schmidt procedure to get an orthonormal basis $(v_1, \dots, v_n)$ of $V$, then $(v_1, \dots, v_k)$ is an orthonormal basis of $U$, and $(v_{k+1}, \dots, v_n)$ is an orthonormal basis of $U^{\perp}$. 
    \item Deduce that $$\dim U + \dim U^{\perp} = \dim V.$$
\end{enumerate}

\end{prob}

\begin{prob}
    Let $V= \mathcal P_2(\mathbb R)$ with the inner product $\langle f,g\rangle = \int_0^1 f(x)g(x)dx$. Consider the functional $F:V\to \mathbb R$ given by $F(f) = f(0) + f'(1)$. Find $g\in \mathcal P_2(\mathbb R)$ representing $F$ in the Riesz representation theorem, i.e.\ such that $F(f) = \langle f\mid g\rangle$.
\end{prob}

\end{document}